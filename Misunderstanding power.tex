\section{Misunderstanding power}
\paragraph{Definition:}

\begin{quote}
\textbf{Wikipedia}: Zipf's law states that given some corpus of natural
language utterances, the frequency of any word is inversely proportional
to its rank in the frequency table. Thus the most frequent word will
occur approximately twice as often as the second most frequent word,
three times as often as the third most frequent word, etc. {[}1{]}
\end{quote}

Related formulations, called power laws, model the
\href{http://www2.econ.uu.nl/users/marrewijk/geography/zipf/index.htm}{size
of cities}, and describe
\href{http://pricetags.wordpress.com/2010/10/26/kleibers-law-growth-and-creativity-in-cities/}{energy
use in animals} and social network effects. Creativity and other social
network effects -- like crime -- are more prevalant in large cities.
Power laws also describe the forces governing
\href{http://shirky.com/writings/powerlaw_weblog.html}{online
participation}. But it is easy to forget this.

\paragraph{Problem:} How many times have we been at a conference or
workshop and heard someone say (or said ourselves) ``wouldn't it be
great if this energy could be sustained all year 'round?'' Or in a
classroom or peer production setting, wondered why it is that everyone
does not participate equally. ``Wouldn't it be great if we could
increase participation?'' But participation in a given population will
fall off according to some power law (see Introduction to Power Laws in
\href{http://www.theuncertaintyprinciple.danoff.org/v2i3.html}{The
Uncertainty Principle, Volume II, Issue 3}). It would be an illusion to
assume that everyone is coming from a similar place with regard to the
various literacies and motivations that are conducive to participation.

\paragraph{(Bogus) Solution:} It can be tempting to adopt a ``provisionist''
attitude, and say: ``If we change our system we will equalize
participation and access.''

\paragraph{Challenges:} Power laws are an inherent epiphenomenon of network
flows. If you can adjust the way the way the network is shaped, for
example, through
\href{http://peeragogy.org/practice/moderation/}{moderatation}, then you
may be able to change the ``exponent'' in the power law. But even so,
``equality'' remains a largely abstract notion. Note, also, that
participation in a given activity tends to fall off over time. It's easy
to imagine writing a hit song or a best selling novel, but hard to pull
this off, because it takes sustained effort over time. See the
anti-pattern
\href{http://peeragogy.org/antipatterns/magical-thinking/}{Magical
Thinking}.

\paragraph{What's Next:} As Paul Graham wrote about programming languages
-- programmers are typically ``satisfied with whatever language they
happen to use, because it dictates the way they think about programs''
-- so too are people often ``satisfied'' with their social environments,
because these tend to dictate the way they think and act in life.
Nevertheless, if we put our minds to it, we can become more ``literate''
in the patterns that make up our world and the ways we can effect
change.

\paragraph{References:}

\begin{enumerate}
\item
  \href{http://en.wikipedia.org/w/index.php?title=Zipf\%27s_law\&oldid=575709945}{Zipf's
  law}. (2013). In \emph{Wikipedia, The Free Encyclopedia}.
\item
  Graham, P. (2001). \href{http://www.paulgraham.com/avg.html}{Beating
  the averages}.
\end{enumerate}
