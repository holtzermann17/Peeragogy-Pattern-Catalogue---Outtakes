\section{Roles}
\paragraph{Definition:} Educational interactions tend to have a number of
different roles associated with them. Everything could bifurcate from
the ``autodidact'', as in, (1) Autodidact, (2) Tutor-Tutee, (3)
Tutor-Tutee-Peer, etc., until we have bursars, librarians, technicians,
janitors, editors of peer reviewed research journals, government policy
makers, spin-off industrial ventures and partnerships, and so on.

\paragraph{Problem:} Even the autodidact may assume different roles at
different points in time - sometimes building a library, sometimes
constructing a model, sometimes checking a proof. The decomposition of
``learning'' into different phases or polarities could be an endless
theoretical task. The simpler problem is to be aware of the roles that
you and your teammates have in the projects you're working on.

\paragraph{Solution:} We've described some exercises on
``\href{http://peeragogy.org/can-we-work-together/}{metacognition}''
that you can apply when thinking about the roles that you're taking on
and those that you'd like to take on in your project.

\paragraph{Challenges:} Roles are often present ``by default'' at the start
of a learning process, and that they may change as the process develops.
Both of these features can be challenging, but they also present
learning opportunities.

\paragraph{What's Next:} We've listed some of the roles for which we're
seeking volunteers in
the \href{http://peeragogy.org/peeragogy-org-roadmap/}{Peeragogy.org
Roadmap}: Volunteer Coordinator, Seminar Coordinator, Usability Guru,
Activities Master, and Tech lead. As with everything else in the
roadmap, this list should be reviewed and revised regularly, as the
roles are understood relative to the actual happenings in the Peeragogy
project.
